%&paper_preamble
\endofdump

\begin{document}

\begin{abstract}
\noindent
Data Compartmentalisation for CHERI
\end{abstract}

\maketitle

\section{Introduction}
Start

\section{Background}
\begin{itemize}
    \item CHERI
    \item compartmentalisation
\end{itemize}

\section{Compartmentalisation Methods and Approaches}

\begin{itemize}
    \item Discuss PCC versus DDC compartmentalisation
    \item Discuss Morello Executive vs Restricted mode, including compartment-in-compartment issue
    \item Propose some uses for a few particular compartmentalisation methods (dunno which for now, think of a few)
    \item General ``embarassingly compartmentalisable'' examples: browser tabs, tabs in general, external libraries (the stuff that Cambridge are doing with their linker), memory allocator, secrets
\end{itemize}

What does compartmentalisation mean? Well, we can think of it as a method of
separating two spaces. In computing, processes are a good example of
compartmentalisation: two distinct programs that (ideally) cannot interact with
one another (except through determined pathways via the kernel), and generally
should be unaware of one another. Compartmentalisation then becomes useful to
protect the system and other processes from a potential rogue or captured
process. Further, how is compartmentalisation defined? Well, this is likely up
to the kernel, to ensure virtual memory spaces do not leak, and hardware
resources are utilised appropriately. There is, to our knowledge, no
one-size-fits-all implementation\andrei{Double check}.

Now, what does compartmentalisation mean in CHERI, or rather from the prism of
CHERI capabilities? A definition which, we believe, encompasses both the intent
behind what compartmentalisation is used for, as well as its usage in both
CHERI purecap and hybrid modes, is \textbf{``the union of all memory spaces
reachable by all capabilities in use at a given time''}. What this means is,
during program execution, if we pause execution of a CHERI program, and look at
all capabilities in registers, then at all capabilities stored in memory
accessible via those capabilities in registers, then repeat until a fixed
point, we can say the union of all those reachable memory regions is the
current compartment. Normally, in a default CHERI hybrid execution, where there
are no explicit capabilities, the address space is entirely reachable, due to
the default, unbounded capabilities installed in the \pcc~and \ddc~registers.
In a purecap CHERI execution, the intrinsic compartment is very restricted, as
the bounds of the current \pcc~capability will only refer to the current
instruction, and then we need to compute the transitive closure\andrei{is this
the right word?} of any capabilities in other registers.

\section{Implementation}
\begin{itemize}
    \item Generic library, attachable to any existing source code, with no required source code changes
    \item Simple API for using compartments, abstracting away all CHERI tidbits
    \item Decide on a manager model, potentially could have a ``mutually distrusting compartments'' model, with in-memory sealed capabilities
    \item Ended up writing a runtime loader for ELF binaries
    \item Discuss design choices, TLS handling, duplicate \texttt{so} files in compartments, \texttt{malloc} interception
    \item Discuss optimisation possibilities, such as using existing loader as basis, using existing \texttt{malloc}, one compartment per so for less memory usage
    \item Mention potential performance gain from using ``compartments as processes''
\end{itemize}

We aimed to write an API for compartmentalisation in the form of a dynamic
library. An end-user could link our API against their code to gain access to
generic compartmentalisation actions, such as create a compartment, execute it,
and destroy it. When thinking at what level we should compartmentalise, the
answer was: to begin with, as simple a level as possible. To us, this meant
compartmentalising against pre-compiled static binaries, as they would
essentially be a blob of stand-alone executable code which we can load into
memory, without requiring any external linkage or transitions to work. While
this seemed to be the easiest choice of compartmentalisation granularity, we
did not realise at the time that we would embark on a journey to implement a
runtime loader.

\andrei{insert static binary rant here - fixed compile-time loading offset, lack of support for static PIE binaries in FreeBSD}

\section{Early Evaluation}
\begin{itemize}
    \item Try to evaluate system \texttt{rtld} performance versus our loader (this will likely be very small, maybe nanoseconds - try on a slow VM?)
    \item Show evaluation results of Hello World lua script (this is the one where we aimed, and had 2x speedup, but only 20% after PR
    \item Evaluate from both security and performance aspect
    \item Security points
    \begin{itemize}
        \item we introduce another single-point of failure, the compartment manager
        \item security dependent on CHERI sealed capabilities
        \item mention issue with using PCC as DDC (I haven't checked if removing the ``use system registers'' permission fixes this)
        \item each compartment overhead should be 2 capabilities worth of memory usage
    \end{itemize}
\end{itemize}

\section{Further Work}
\begin{itemize}
    \item Performance improvements, especially using a useful malloc
    \item Consider merging with system \texttt{rtld}
    \item Expand to PCC compartmentalisation (should be straightforward, just need to work on the assembly part mostly, or even entirely)
    \item Implement compartmentalisation in an actual tool (a browser like w3m, or a VM like Lua, something straightforward)
\end{itemize}

\section{Conclusion}
\begin{itemize}
    \item Proof of concept for CHERI hybrid, using DDC for compartmentalisation
    \item Idea using compartments as processes
    \item Initial performance poor, but we overcome the CHERI hardware performance overhead at least
\end{itemize}
End\cite{sison95simultaneous}

\bibliographystyle{ACM-Reference-Format}
\bibliography{bib}

\end{document}
